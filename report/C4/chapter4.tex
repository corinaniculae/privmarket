\chapter{Project Evaluation}

This chapter contains the evaluation methods that are going to be used in order to measure the quality of the project's contributions. The following criteria will be employed:
\begin{enumerate}
\item \textbf{completeness of the achieved location privacy} will be used in order to make sure the proposed solution covers all possible threats that lead to exposing the smartphone users' identity.

\item \textbf{smartphone performance} will be measured in order to make sure that the data collection and storage, as well as the data computations, will not impose an overhead that reduces the overall performance of the smartphone.

\item \textbf{users' perception of the incentive} might represent an intriguing metric to be put into practice, but the project would like to get a better insight on the incentive system that it is proposing.
\end{enumerate}

The following sections discuss the possible methods and approaches that are considered in order to have a quality scale system and to assign evaluation metrics to the previously mentioned criteria.

\section{Completeness of the Achieved Location Privacy}
According to Shokri \& co. \cite{quantifylocationprivacy}, the threat models that support the current solutions do not guarantee completeness. In order to address this issue, the paper provides a framework that can be used to quantify the location privacy of the project. Currently, this framework is considered in order to test and reason about the completeness of the location privacy that the current project will achieve.

\section{Smartphone Performance}
Given the fact that a decentralised system will be implemented, by attaching an individual collector to each smartphone user, it will be sensible to check that the running computations will not add an overhead that cannot be neglected. In this sense, the smartphone performance will be measured (mainly battery and bandwidth usage), in order to better understand what operations can be run on the smartphone vs. remotely, or online vs. offline.

\section{Users' Perception of the Incentive}
Lastly, this criteria will give a better grasp of the current view of the smartphone users on the privacy and anonymity topic. Since this is an extremely subjective criteria, the project will seek feedback from a focus group that will be asked to try out the application over a short period of time and, at the end, the evaluation study will assess their opinion on it. This assessment will include the actual users' data into a cluster of virtual users in order to also test their visibility (or better said, the lack of it), within the system. The focus group will know that at the end of the study one of them might be entitled in 'winning the lottery' of the system (symbolically), and the assessment will also ask for their opinion on this aspect of the project.

Another alternative, given less available time, will be to simply research the possible users' opinion on being part into such a system, on a theoretical level. In that case a focus group might just be shown a demo of the proof of concept and collect their opinions afterwards, as part of a single meeting session.